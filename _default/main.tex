\documentclass[dvipdfmx]{beamer}

\usepackage{amsmath, amssymb}
\usetheme{Madrid}% Sets basic formatting.  Lots of options, google "beamer themes"
\usecolortheme{dolphin}% Sets the colour scheme.  Lots of options, google "beamer color themes"

\usefonttheme{professionalfonts}
\setbeamertemplate{navigation symbols}{}
\setbeamercolor{normal text}{bg=black!3}
\setbeamertemplate{frametitle}[default][center] % Sets the frametitle center

% --- page number ---
% \setbeamertemplate{footline}{%
% 	\raisebox{10pt}{\makebox[\paperwidth]{\hfill\makebox[7em]{\normalsize\texttt{\insertframenumber/\inserttotalframenumber}}}}%
% }

\date{}	% Insert the date of your presentation. \today gives an unsurprising automatic date.
\title[Template]{日本語もおk?}	% Insert your title.  Depending on the theme you choose above, a "short title" might be useful, as it will appear on the footer of each slide.
\author[Y Hikima]{Yasunari HIKIMA} % Insert your name
\institute[Fujitsu Lab]{Fujitsu Research} % Self-explanatory

% Presenter's note
% \setbeameroption{show notes on second screen}

\title{Beamer Example}
\author{eqs}
\date{\today}

\begin{document}

\begin{frame}[plain]
    \maketitle
\end{frame}

\begin{frame}{Uniform convergence}	

\begin{definition}	% There are lots of "theorem-like" environments for beamer just as usual with LaTeX: definition, theorem, lemma, example, proof, etc...

A sequence of functions $f_n \colon \mathbb{R} \to \mathbb{R}$ \emph{converges uniformly} to a function $f \colon \mathbb{R} \to \mathbb{R}$ if for all $\epsilon > 0$ there exists an $N \in \mathbb{N}$ such that $n \geq N$ implies 
\[
\sup_{x \in \mathbb{R}} |f_n(x) - f(x)| < \epsilon.
\]

\end{definition}

\pause	% Generates a break in the slide presentation

\begin{block}{Pointwise and uniform continuity} % Blocks are a beamer speciality.

\begin{itemize}
\item Uniform convergence implies pointwise convergence
\item Pointwise convergence does not imply uniform convergence
\end{itemize}

\end{block}

\pause

\begin{theorem}
Let $f_n \colon\mathbb{R} \to \mathbb{R}$ be continuous and converge uniformly to $f \colon \mathbb{R} \to \mathbb{R}$.  Then $f$ is continuous.
\end{theorem}

\end{frame}


\end{document}
